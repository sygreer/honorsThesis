\title{Introduction}
\author{Sarah Yvonne Greer}
\maketitle

%Data matching problems are prevalent in many applications of seismic data processing.
Data matching is a conceptually simple problem that is present in many computational geophysics applications. 
Essentially, given two data sets, data matching can be thought of as finding the transformation from one data set to the other. 
This is useful in applications such as multi-component seismic image registration, where two separate seismic images, one P-wave image and one S- wave image, are acquired over the same area. 
These must be properly matched, as the two images are not temporally aligned and have different frequency and amplitude content, in order for them to be directly compared and interpreted. 

Many other examples of data matching are prevalent in geophysical applications, such as timelapse image registration, seismic and well-log tying, deconvolution, and surface-related multiple elimination.

In this thesis, I will outline several methods and applications of seismic data matching, and introduce a new method for matching frequency content between data sets.


%However, many data matching algorithms, such as Dynamic Time Warping, can be computationally expensive, especially when applied to large multi-dimensional seismic data sets.
\subsection{Background}
Seismic data contains fundamentally non-stationary variations in attributes such as frequency and amplitude content.
